\documentclass[a4paper,12pt]{letter}

% Some of the article customisations are relevant for this class

\name{Dr. Grant Wythoff} % To be used for the return address on the envelope
\signature{Dr. Grant Wythoff} % Goes after the closing (ie at the end of the letter, with space for a signature)
\address{Grant Wythoff \\ Postdoctoral Fellow \\ Society of Fellows in the Humanities \\ Lecturer, Department of English \\ Columbia University \\ \href{mailto:grant.wythoff@gmail.com}{grant.wythoff@gmail.com} \\ 856-296-9103}
% Alternatively, these may be set on an individual basis within each letter environment.

\makelabels % this command prints envelope labels on the final page of the document

\begin{document}
\begin{letter}{Editor \\ University of Minnesota Press \\ Minneapolis, MN}

\opening{Dear Editor} % eg Hello.

I write to propose for publication a critical edition of essays by Hugo
Gernsback (1884-1967) titled \emph{The Perversity of Things: Writings on
Media, Technology, and Science Fiction.} This volume makes available
foundational texts by the magazine editor, inventor, and novelist that
have been out of print since their original publication in his magazines
from 1905-1933. The evolution of the Gernsback publications over this
period -- from an electrical parts catalog into a fully-fledged literary
genre -- is an untold story in American literary and media history.

While Gernsback is best remembered for launching the first science
fiction magazine in 1926 (\emph{Amazing Stories}), and his name adorns
the award given out each year to the best works in the genre (the Hugo
Award), he now receives little more than a brief mention in science
fiction studies. The incredibly rich context of the genre's birth in
Gernsback's fleet of technical publications for the amateur
experimenter, as well as his work as a pioneer in media technologies and
broadcasting techniques, have been entirely overlooked.

In these publications, one could find a literary treatise on what the
genre of ``scientifiction'' should look like alongside blueprints for a
homebrewed television receiver well before its commercial possibility.
\emph{Electrical Experimenter} and \emph{Modern Electrics} published
profiles of new technologies used in cinema production and projection (a
medium itself in its infancy), commenting once those films were released
on the depiction of new and often imaginary devices used by the
characters on screen. Long before Gernsback founded \emph{Amazing
Stories,} these magazines used speculative fiction to find a language
suited to the analysis of emerging media like radio, television, or the
more exotic osophone and telegraphone.

\emph{The Perversity of Things,} named after a Gernsback essay on the
influence that objects exert on thought, seeks to provide a reappraisal
of both the ``hard'' technical roots of American science fiction and the
highly speculative orientation toward media technologies during this
period. Science fiction in its infancy wasn't just a literary form, it
was a mode of interacting with new media. The need for this volume has
become apparent in recent years as we forge new technological
revolutions in everyday life, a paradigm shift similar to the one that
brought readers to the Gernsback publications a century ago.

This volume has the potential for broad-based appeal to both academic
and popular audiences. A scant few works of scholarship have been
devoted to this period in the history of science fiction, and it is my
hope that the availability of these texts can significantly revise our
understanding of the genre's origins. Because Gernsback's writings
explore many branching paths not taken in the history of media
technologies, this book will be useful for media studies as well, a
field increasingly interested in the ``archaeology'' of dead, hybrid,
and imaginary media. In design studies, where ``design fiction'' and
``speculative design'' are becoming vibrant fields of inquiry, a book
surveying the prehistory of rapid prototyping and tinkering would prove
a valuable resource. These essays can find an audience outside of
academia as well, as they were inherently meant to be read by a popular
audience without sacrificing the sophistication of their ideas.

The book will be organized chronologically in order to highlight the
conceptual fluidity of these writings that moved with ease from
literature to technology. I will, however, make available a thematic
table of contents and organize my introduction around the following five
areas:

\begin{itemize}
\itemsep1pt\parskip0pt\parsep0pt
\item
  Tinkering: reports on broadcast experiments at Gernsback's WRNY radio
  station, new inventions developed at the Electro Importing Company
  labs, and critical essays on techniques for thinking creatively
  through technology
\item
  Scientifiction: foundational writings on science fiction as a distinct
  genre, including literary historical essays on writers Gernsback
  understood to be predecessors: Wells, Verne, Poe, Luis Senarens,
  Clement Fezandié
\item
  Media History: later works that read current inventions in light of
  their not so distant precursors, profiling forgotten (and often
  quirky) paths not taken in the development of radio and television
\item
  Broadcast Regulation: activism, community organization, and manifestos
  written in service of radio amateur rights after the US government
  banned all public wireless activities in the wake of World War I;
  includes later writings for the technocracy movement
\item
  Selected Fiction: four short stories and the original serialized
  version of the famous \emph{Ralph 124C 41+}, all of which have been
  out of print since their original magazine publication
\end{itemize}

To offer you a sense of this material I have enclosed two sample
articles, using a page layout mockup I envision with wide footnote
columns that can include (public domain) images and blueprints as well.
I have been in contact with Gernsback's family, who is very enthusiastic
about this project and have assured me that these magazines remain in
the public domain.

I have previously published on Gernsback in \emph{Grey Room} and
\emph{Wi: Journal of Mobile Media,} as well as presented material from
this book at Columbia, Princeton, the University of Pennsylvania, the
Winterthur Museum of American material culture, and Université de Paris
XIII, among other venues.

Drawing on extensive archival research from the collections at the Hugo
Gernsback Papers at Syracuse University (research generously funded by
the Princeton Program in American Studies), I propose a selected edition
of roughly 80 essays ranging from 500 to 2,500 words each. This edition,
including editorial apparatus and footnotes, would run approximately
100,000 words.

Minnesota's \ldots{} series seems a strong fit for my work, given your
commitment in recent publications to\ldots{}

I am currently a Postdoctoral Fellow in the Society of Fellows in the
Humanities at Columbia University, where I am developing my dissertation
on the cultural history of the gadget into a book manuscript (see my
enclosed CV). Please do not hesitate to contact me if you have any
questions. I have submitted this query to two other presses, and I will
let you know immediately should I receive interest from one of them. In
the meantime, I look forward to hearing from you.

\closing{Sincerely,} % eg Regards,

\cc{} % people this letter is cc-ed to
\encl{CV, Book Proposal, Table of Contents, Two Annotated Essays} % list of anything enclosed
\ps{} % any post scriptums. ``PS'' labels must be put in manually

\end{letter}
\end{document}
