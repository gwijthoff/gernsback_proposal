% Gernsback Book Proposal v.1

\documentclass{article}
%\usepackage[utf8]{inputenc}
\usepackage{hyperref, fancyhdr} %hyperlinks; header text
\usepackage[absolute]{textpos} %allows for floating text anywhere, incl. in frontmatter

\usepackage[style=verbose-trad2,natbib=true]{biblatex}
\bibliography{gernsback} %chooses .bib file
\defbibheading{bibliography}{\section*{Bibliography}} %to have separate "bibliography section"

\pagestyle{fancy} %adds header
\lhead{Wythoff}
\chead{}
\rhead{Proposal: \textit{The Perversity of Things}}

\title{ \textit{The Perversity of Things:} \\ 
{Writings on Media, Technology, \& Science Fiction} \\
Hugo Gernsback}
\author{Grant Wythoff, editor \\
Society of Fellows in the Humanities \\
Columbia University \\
\url{http://wythoff.net}
}
\date{\today}

\begin{document}

\begin{textblock}{5}(3.5,1)
\noindent\Large Book Prospectus
\end{textblock}

\maketitle

\section{Introduction}

In the April 1911 issue of Hugo Gernsback's \textit{Modern Electrics}, a fifteen-year-old Lewis Mumford published his first piece of writing in the magazine's "Experimental Department," a place for readers' home-brewed designs for novel electrical devices.  Best known for the magisterial history of technology \textit{Technics and Civilization} (1934), the young Mumford describes here his portable wireless telegraph receiver:  "smaller than a small camera, in which sensitiveness is not sacrificed to saving of space."  Flipping back three pages in the same issue, one can find the first installment of Gernsback's serial novel \textit{Ralph 124C 41+: A Romance of the Year 2660}, one of the earliest works of modern science fiction. \autocite{mumford_portable_1911}

�who later cited his experiences with these magazines as formative for his mature theorization of "technology's spiritual contribution to culture"

The proximity of this future media theorist's first published piece of writing to one of the foundational works of science fiction is no coincidence.  Mumford's early preoccupation as a tinkerer took root in a forum for the exchange of designs, components, and visions of the future.  Gernsback's \textit{Modern Electrics} began as a mail-order catalogue for imported wireless parts and exotic electrical miscellany in 1905, gradually evolving into a fleet of popular and specialist magazines, including \textit{Electrical Experimenter, Radio News, Science \& Invention}, and \textit{Technocracy Review}.  One could find in these publications a literary treatise on what the genre of "scientifiction" should look like alongside a blueprint for a homemade Nipkow disk television set.  Translations of an influential German handbook titled \textit{The Practical Electrician} ran next to speculative articles on what it would take to transmit wireless signals to (and from) Mars.

Though Hugo Gernsback is best remembered today for launching the first science fiction magazine, \textit{Amazing Stories} (1926), and the Hugo Award is given out each year to the best works in the genre, he now receives little more than a one- to two-sentence nod in science fiction studies.  He is associated with the gaudy covers of his magazines and a "crude and heavy-handed" editorial style that perpetuated many of the negative stereotypes still associated with science fiction today.  Much of this attitude has been inherited from a generation of science fiction scholars who were not academics but editors themselves, and disparaged Gernsback's editorial practices as well as the infamously low wages he paid his writers.  But this inherited version focuses only on the period from \textit{Amazing Stories} and after, entirely overlooking the context of the genre's birth in Gernsback's fleet of electrical experimenter magazines, as well as his work as a pioneer in media technologies and broadcasting techniques.

What sets Gernsback's magazines apart from other technical publications of the period like \textit{The Wireless Age} and \textit{QST} is the way in which he encouraged his contributors to enlist fiction in their attempts to find a language suited to the analysis of emerging media such as radio, television, or the more exotic osophone and telegraphone.  For instance, when readers submitting a design encountered difficulties in describing its specifications -- perhaps a certain component was technically or economically unfeasible -- they would stitch their incomplete engineering diagrams together with narrative threads.  With his monthly editorials, feature articles, and short fiction, Gernsback pioneered a kind of writing that combined hard technical description with an openness to the fantastic.  It was a mixture out of which an entire literary genre emerged to tackle the question of the distinctive specificity of "medium" in a new wireless age.

For Frank Pichler, "Hugo Gernsback is the father of American electronic culture."

There was a seamless transition from technical experimentation to narrative speculation.  Science fiction wasn't just a narrative phenomenon, it was a mode of technical interaction.

\textit{The Perversity of Things}, named after a Gernsback essay on the influence that objects exert on thought, provides the first critical edition of Gernsback's media historical, literary critical, and technical writings.  This collection will occasion a reappraisal of both the "hard" technical roots of American science fiction and the highly speculative orientation toward media technologies in the period.  \textit{The Perversity of Things} makes available foundational texts in media history and science fiction that have been out of print since their original magazine publication from 1905-1933.  Through extensive archival research, I provide a new picture of modern science fiction as a genre that emerged out of an electrical supply catalogue.  

The literary historical gambit of this book is to recover the radical sense of openness that greeted not only the basement tinkerer working through the feasibility of transmitting images over a wire, but also the author of "scientifiction" stories who possessed a highly sophisticated awareness of the fact that "Two hundred years ago, stories of this kind were not possible."  Often, these individuals were one and the same, weaving together functional and fictional devices in a manner that served for them as a form of scientific discovery in itself.

%What does it mean for technology to render new narrative forms possible?  How do we register these formal shifts?  What were the circuits through which a literary community came to understand the incursion of new technologies, before 

\section{Background}

\subsection{Gernsback's biography}

Born Hugo Gernsbacher in Luxembourg in 1884, the son of a successful wine wholesaler, Gernsback immigrated to the United States in 1904 at the age of 19.  Carrying with him the design for a new kind of dry cell battery, Gernsback published his first article on the device a year later in \textit{Scientific American} under that most American of names, "Huck."  Gernsback sold the battery patent to the Packard Motor Car Company, who ended up using technology in their ignition systems.  

With the profits of his sale, Gernsback formed the Electro Importing Company, an importer of specialized electrical equipment from Europe and the first mail-order radio house in the country.  Through their catalog and retail store at 84 West Broadway in New York, the company provided access to specialized wireless and electrical equipment not found anywhere outside of Europe.  Electro Importing catered to a diverse clientele, providing their more advanced basement experimenters with the first vacuum tube offered for sale to the general public in 1911, and manufacturing for their novice users in 1905 the first fully assembled radio set commercially available, the Telimco.

After several issues of their mail order catalog and a growing subscription list, Electro Importing began including features, editorials, and letters to the editor.  Between 1906 and 1910, the catalog grew into a series of monthly magazines for the wireless homebrewer, beginning with \textit{Modern Electrics} in 1908 and the offshoot Experimenter Publishing Company in 1915.  While \textit{Modern Electrics} still advertised the equipment Electro Importing offered for sale in a familiar grid layout with ordering instructions, it also included feature articles detailing the latest research into experimental media technologies in America, Germany, France, and in Gernsback's own company offices.  Regular reporters and celebrity guest contributors like Lee De Forest, Thomas Edison, and Nikola Tesla, provided reports on television, wireless telephony, and the use of novel electrical apparatuses in film and theatrical productions, each of which would go into a great degree of technical detail.

But the hallmark of the magazine and those to follow in its wake became its more speculative articles, those that were willing to extrapolate fantastic scenarios out of the technical details at hand.  In "Signaling to Mars," Gernsback detailed the conditions that would have to obtain in order for Earth to send messages via wireless telegraph to the red planet.  For readers of \textit{Modern Electrics}, the technical context in which this highly speculative article appeared only led credence to the idea that contact with an alien civilization was right around the corner.  In the copy of this issue at Princeton University's Firestone Library, someone inserted a newspaper clipping (now a permanently affixed leaf within the bound volume) that tells of a new distance record for wireless signaling, from San Francisco to the Pacific Mail Line steamship Korea as it made its way across the ocean.  Left there as if to vouch for the plausibility of the idea that we'll soon be able to connect with our nearest planetary neighbor, the clipping provides a wonderful sense of how it was people read these magazines.  Though the Gernsback titles eventually became infamous for their sometimes outlandish claims -- that electric current might clean us better than water; that the success of a marriage can be predicted using gadgets assembled out of various household supplies -- they were always presented through a lens of supposedly scientific rationality.  This frame affected the reception of the magazines by their readers, the design ethos that grew up around them, and the kind of fiction they eventually produced.

Internationalism�

The shape of media to come took on an iconography all its own through the illustrations of Frank R. Paul.  Paul's depictions of gadgetry circulated widely beyond their original publication venues in a way that has never before been given any attention.  Plans for the osophon, a device Gernsback designed to replace headphones by transmitting sound through vibrations in the jawbone of the listener, were published and reviewed in the German journal Der Radio-Amateur.[1]  Paul's sketch of a man using a tuning fork to calibrate the speed of the 1928 Science and Invention Nipkow disk television receiver was republished the following year in the Chinese film journal Yingxi zazhi (Shadow Play Magazine) as an illustration of recent research into television, what was referred to in the article as, directly translated, "wireless cinema."[2]  Further research into the global circulation of these images could reveal what we might think of as an emergent, transnational media culture in the mid- to late-1920s that thrived off discussions of recombinant media technologies and dreams of a media saturated future.
?
[1] Eugen Nesper, "Das Osophon von H. Gernsback," Der Radio-Amateur (vol. 2, no.1, 1924), 10.
[2] Weihong Bao, "Sympathetic Vibrations:  Hypnotism, Wireless Cinema, and the Invention of Intermedial Spectatorship in 1920s China," Media Histories: Epistemology, Materiality, Temporality, Columbia University, New York, 26 March 2011.  Bao located Paul's illustration of the Science and Invention television receiver in Shen Xiaose ???, "Dianyingjie de jiizhong xin faming" (several new inventions in the film world), Dianying yuebao (vol. 9, 1929), 1-64.  The term for television used in this article is "wuxian dianying ???? (wireless cinema, or, more literally, wireless electric shadow, or radio shadow)."  Paul's images were originally published in "Radio Movie," Science and Invention, vol. 16 no. 7 (November 1928), p. 622-3.

\section{Gernsback's reputation in the critical literature}

The name Hugo Gernsback has become something of a dirty word in science fiction studies, the sound itself enough to drive off any further inquiry.  Among his worst detractors are Brian Aldiss and Brian Stableford, the latter of whom sees Amazing Stories as little more than a marketing gimmick for Gernsback's technical ventures.  Pointing out that stories of technical wizardry are conveniently placed alongside advertisements for the Electro Importing Company and WRNY broadcasts, he argues that the fiction in Amazing was merely another means of spreading enthusiasm for the devices sold by Gernsback's company.[1]  For Aldiss, who agrees that the magazine "exists as propaganda for the wares of the inventor," Amazing Stories displays a "highly technological romanticism" that was at best nostalgic and at worst fascist.  "It is easy to argue that Hugo Gernsback was one of the worst disasters to ever hit the science fiction field.  Not only did the segregation of science fiction into magazines designed especially for it, ghetto-fashion, guarantee that various orthodoxies would be established inimical to a thriving literature, but Gernsback himself was utterly without any literary understanding.  He created dangerous precedents which many later editors in the field followed."[2]  When critics such as this do give Gernsback some credit in the foundation of modern American SF, it is for setting up a forum in which it could take shape as a distinct genre.  As Alexi and Cory Panshin write, "In the pages of Amazing, SF literature at last became identified by a single name�'scientifiction.'  It was provided with a history.  It was defined and demonstrated.  It was consolidated and unified.  In Amazing, SF became conscious of itself."[3]  When Hugo Gernsback is credited as a founder of science fiction, it is as a genre rather than any particular literary technique.  According to John Clute and Peter Nicholls, Gernsback "gave the genre a local habitation and a name," and for James Gunn, he "provided a focus for enthusiasm, for publication, for development," and bestowed science fiction with its "characteristic content, a characteristic form, and characteristic purposes."[4]
	These critical histories (each of which are written by SF novelists in their own right) proceed as if propelled by their own fantastic, alternate history:  what if science fiction left us with texts as highly valued as the works of high modernism from the very beginning?  No doubt the field of science fiction studies has done a great deal of work in order to elevate certain works of the genre (mainly from the 1960s and 70s New Wave) to a kind of academic respectability.  But these assumptions about the genre's magazine era beginnings have never been questioned, nor have any of Gernsback's publications save for Amazing Stories been submitted to any kind of rigorous analysis.  The editorial function many SF histories begrudgingly ascribe to Gernsback overlooks the much more interesting heritage of magazine SF within a wider tradition of thinking about new media. 

?
[1] Brian Stableford, "Creators of Science Fiction 10: Hugo Gernsback," Interzone (no. 126, December 1997), 47-50.  Brian Stableford, "Science Fiction Before the Genre," The Cambridge Companion to Science Fiction, ed. Edward James and Farah Mendlesohn (Cambridge University Press, 2003), 30.  "Gernsback's devotion to the rhetoric of hard science, of which Ross makes much, had little to do with the stories he accepted," writes Istvan Csicsery-Ronay, Jr., "Postmodern Technoculture, or the Gordian Knot Revisited: Jameson's Postmodernism, Ross's Strange Weather, and the Penley-Ross Technoculture," Science Fiction Studies (vol. 19, no. 3, November 1992).
[2] Brian W. Aldiss, Trillion-Year Spree: The History of Science Fiction (London: Victor Gollancz, Ltd: 1986), 175, 202.  A rewritten edition of Billion-Year Spree.
[3] Alexi Panshin and Cory Panshin, The World Beyond the Hill: Science Fiction and the Quest for Transcendence (Los Angeles: Jeremy P. Tarcher, Inc: 1989), 170.
[4] John Clute and Peter Nicholls, The Encyclopedia of Science Fiction (London: Orbit Books, 1993), 491. James Gunn, Alternate Worlds: The Illustrated History of Science Fiction (A and W Visual Library, 1975), 128.

========

The canonical story about Hugo�Gernsback is that he launched the genre of science fiction as the founding editor of \textit{Amazing Stories}�in April of 1926.��Gernsback�treated the magazine as merely a commercial venture, wrote in a "crude and heavy-handed" style, and now usually receives little more than a cursory, one-sentence nod in critical works on the genre.

This story has held sway over the reception of Gernsback until very recently.  A number of books and museum exhibits are beginning to rethink Gernsback's role in media history, and the origins of science fiction more broadly.

\subsection{Recent books}

Craig Yoe's \textit{The Best of Sexology: Kinky and Kooky Excerpts from America's First Sex Magazine} (Running Press, 2008) is an anthology of articles from one of Gernsback's more adventurous titles, \textit{Sexology}.  Unfortunately, the volume does little more than reproduce page scans of articles from the magazine, 

\subsection{Recent exhibits}

\section{Summary of Book / Thematic Outline}

The evolution of these publications, as well as their regularly contributing readership of amateur tinkerers and fiction writers, from an electrical parts catalog into a fully-fledged literary genre is an untold story in American literary and media history.  My introduction will situate Gernsback's essays within this context.

The essays will proceed chronologically rather than thematically, allowing readers to experience the � soup out of which 

A thematic index in the back of the book will provide a further entry point for readers interested in particular aspects of Gernsback's thinking.  These sections are explained below.

\subsection{Science fiction}

Today, the phrase "science fiction" conjures up images of bug-eyed monsters, ray guns, starships, and sonic screwdrivers.  But in the opening decades of the twentieth century, before a century's accretion of images, narratives, and clich�s, that which was not yet called science fiction consisted of a great number of concrete practices all geared toward a reckoning with the technological revolutions in the fabric of everyday life.  "Science," wrote Gernsback in the inaugural issue of Amazing Stories, 
\begin{quote}
through its various branches of mechanics, astronomy, etc., enters so intimately into all our lives today, and we are so much immersed in this science, that we have become rather prone to take new inventions and discoveries for granted. Our entire mode of living has been changed with the present progress, and it is little wonder, therefore, that many fantastic situations [...] are brought about today. It is in these situations that the new romancers find their great inspiration.\autocite{gernsback_new_1926}
\end{quote}

As early as 1915, Gernsback cites dime novels like Deadwood Dick and authors like Luis Senarens, revealing an awareness of the genealogy of science fiction over a decade before the launch of \textit{Amazing Stories} in 1926, when most SF critics locate the birth of the genre.
%Informal modernism stuff? or does this play my hand too soon, give away too much from the Gadgetry book?

\subsection{Tinkering and technology}

Experiments were conducted and reported on at Gernsback's radio station WRNY with broadcast media and the effects of various instruments and signal processing techniques on the auditory perception of the station's listeners.  

In one editorial for Radio News, Gernsback describes the television as just an add-on or expansion kit to a normal domestic radio set.  "I am quite certain, for one, that the final television apparatus on your radio set will take up no more room than your present cone speaker."  On August 13, 1928, apparently as a joint promotional stunt for the station and "televisor" receivers designed by the Pilot Electrical Company, WRNY became "the first standard radio station to transmit a television image (the face of Mrs. John Geloso)."  A 1.5 square inch image of Mrs. Geloso's face "was enlarged by a magnifying glass to three inches so it could be viewed by 500 persons at Philosophy Hall at New York University."  In these experimental television broadcasts, sight and sound signals were sent out alternatively rather than simultaneously, a technique known as interleaving.  "Viewers would first see the face of a performer and a few seconds later would hear the voice.  Gernsback used a system designed by Theodore Nakken with receivers built by the Pilot Electrical Company for the broadcasts.  The performances took place for 5 minutes every hour."[1]  

Though this goes on record as one of the earliest public television broadcasts, the Nipkow disk model was by that time decidedly low tech compared to the cathode ray tubes being pioneered by Philo Farnsworth and Vladimir Zworyin.
?
[1] Keith Massie and Stephen D. Perry, "Hugo Gernsback and Radio Magazines: An Influential Intersection in Broadcast History," Journal of Radio Studies (vol. 8 no. 2, 2002), 277.

Rather, he had an abiding interest in the aesthetics of broadcasting audio, which he experimented with at his radio station WRNY, broadcasting concerts of electric keyboards in 1924 and remarking that the instrument's notes "have practically no overtones," or reporting on experiments with the lower and upper limits of human hearing in "audio-frequency oscillator" experiments, for instance.[1]
?
[1] Hugo Gernsback, "The Pianorad," Radio News (vol. 8, no. 5, November 1926), 493.

\subsection{Media history}

In Gernsback's editorials, media history is evoked not merely as a nostalgic trip back to the devices of yesteryear, as it often is today in retro-kitsch, but as an archive of possibilities ripe for future experimentation.  For instance, in a 1927 editorial "Radio Steps Out," at a moment when national broadcast networks were flickering to life and music and variety programs were flooding the country, the medium of radio had become a fixed idea in people's minds that papered over the inherent abilities of the underlying technology.  Looking back on that strange trajectory in which the technology underlying wireless telegraphy became "radio," an everyday part of household furniture, Gernsback writes, "the public at large is not aware of the fact that the art of radio is used for hundreds of different purposes aside from broadcasting and telegraphy."  

"There is hardly any industry today that cannot make use of radio instruments in some phase of its work."

Examples range from a force-field like burglar alarm, to automating the recording of lightning strikes, from measuring the minute weight and touch of a fly to scanning factory workers for stolen metals.  And, in a forgotten example of Gernsback's own from the 1900s (which he christened the "Dynamophone," electric motors can be started remotely by the human voice, proving that "the apparatus foreshadowed broadcasting: the human voice actually did create effects at the receiving end," both for machines and humans.

Gernsback reminds us that the inherent abilities of wireless, of information transmitted through the air, has now itself been scattered by the winds of technological evolution and inflects our understanding of and interaction with a fantastic number of techniques, technologies, and media.

%similar to the "sentiment" of media archaeology today (Parikka), short-circuiting the commonly accepted trajectories of media history allows for new, creative alignments with forgotten and possible technologies.

\subsection{Broadcast regulation}

With the outbreak of World War I, the U.S. Navy took over all broadcast 

forming "The Radio League of America"

As Tim Wu argues, Radio News even served as one of the first broadcast programming guides in the country's history, publishing lists of each radio station in operation, along with their frequencies and "what one might expect to hear on them�a forerunner of the once hugely profitable TV Guide" (39)

1910, Gernsback forms the Wireless Association of America, "a pioneer organization through which he worked to thwart the U.S. Government and especially the U.S. Navy in their 'stupid short-sighted effort' not to control amateur wireless but to impose wavelength restrictions that could have ruled it out of the air.  Gernsback emerged as the single hero of the heroic fight to preserve amateur wireless.  A thankful U.S. Navy appreciated this a few years later during World War I, when it suddenly needed thousands of wireless operators.  They came mostly from the ranks of the Wireless Association of America which the Navy had tried to wreck."  (autobio 39)

"The fact that Hugo Gernsback greatly affected the writing of the Wireless Act [of 1912] can be seen by comparing parts of his editorial from the February 1912 issue of Modern Electrics to the final contents of the act."

Gernsback led anti-Navy monopoly protests?  Check it out in Barnouw�  Also, Drown only gives part of this story � in a mere 5 or so years, RCA itself threatens becoming a corporate monopoly, hardly a good replacement for a federal one:  "First regulated by the Radio Act of 1912, amateur radio was shut down by executive order for the duration of World War I on April 6, 1917.  In 1918, the government began to consider legislation that would significantly reduce interference on the airwaves by giving a monopoly to the United States Navy.  Amateur protest led by radio promoters Hiram Percy Maxim and Hugo Gernsback won a narrow band enclave on the short waves, but by the end of 1919, the Radio Corporation of America was formed after 'thorough discussion between Rear Admiral W.H.G. Bullard of the US Navy and Owen D. Young of General Electric' (Barnouw, Tube of Plenty, 21).  Established from the patents of General Electric, Westinghouse, ATandT and United Fruit, and presided over by General James G. Harbord, RCA would manage the business and technology of radio in the national interest." (54) @regulation

Kennedy:  "With the war over, the question of "regulating the amateur" came up again. The 65th Congress proposed to amend the Alexander Wireless Bill, but the proposed amend- ments forbade so many things essential to-the amateur that the editor was moved to lampoon the bill in a bitter cartoon in the February 1919 Electvical Experimenter. The bill was killed. The hated demon had been effectively exorcised by the power of the cartoonist's pen and of the printed word. The Acting Secretary of the Navy announced that, effec- tive April 15, 1919, all restrictions were removed on radio receiving stations other than those used for reception of commercial traffic."

1915 May -- "Amateur Wireless Plants Closed By Government."  anon.  p. 28.  http://earlyradiohistory.us/1915ban.htm  A seemingly indiscriminate crackdown on amateur stations�  From website:  " During the first two-and-one-half years of the war the U.S. was officially neutral, and President Wilson assigned the U.S. Navy the task of insuring that U.S. radio stations respected this neutrality. Acting under this authority, for a few months the Navy banned all amateur sending and receiving in the west, as reported in Amateur Wireless Plants Closed By Government in the May, 1915 The Electrical Experimenter, although under the circumstances these restrictions appear to have been somewhat premature and excessive. (In his 1915 annual report, Victor Blue, Chief of the Navy's Bureau of Navigation, noted that "in one naval district all amateur stations were closed... for a time sufficient to impress upon their owners the necessity for keeping the transmission of messages to a minimum.")"

\subsection{Selected fiction}

While several editions of Gernsback's \textit{Ralph 124C 41+} was republished in 1950, 1958, and 2000, these are of a revised version Gernsback reran in 1929.  The original, serial version from 1911 has not been seen since its original magazine print run.  There are significant differences between these two versions of the novel that deserve to be looked at by a wider audience.

Other short stories by Gernsback that have not been reprinted include "New York A.D. 2660" (1911), "The Magnetic Storm" (1918), "The Electric Duel" (1923), and "The Killing Flash" (1929).  As mentioned above, the speculative mode of these fictions is not out of step with the technical articles they were published alongside.  Including them within the continuum of essays on broadcast regulation and tinkering allows them to be experienced in a context similar to their original intentions/publication.

\section{Specifications}

%Word count, illustrations, my introduction, timeline to completion
%I have been in touch with members of the Gernsback family, who have assured me that Gernsback's writings are now in the public domain, and are willing to help with this project in any way that may be needed.

\pagebreak
\printbibliography

\end{document}
